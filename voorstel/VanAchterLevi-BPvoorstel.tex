%==============================================================================
% Sjabloon onderzoeksvoorstel bachproef
%==============================================================================
% Gebaseerd op document class `hogent-article'
% zie <https://github.com/HoGentTIN/latex-hogent-article>

% Voor een voorstel in het Engels: voeg de documentclass-optie [english] toe.
% Let op: kan enkel na toestemming van de bachelorproefcoördinator!
\documentclass{hogent-article}

% Invoegen bibliografiebestand
\addbibresource{voorstel.bib}

% Informatie over de opleiding, het vak en soort opdracht
\studyprogramme{Professionele bachelor toegepaste informatica}
\course{Bachelorproef}
\assignmenttype{Onderzoeksvoorstel}
% Voor een voorstel in het Engels, haal de volgende 3 regels uit commentaar
% \studyprogramme{Bachelor of applied information technology}
% \course{Bachelor thesis}
% \assignmenttype{Research proposal}

\academicyear{2023-2024} % TODO: pas het academiejaar aan

% TODO: Werktitel
\title{Onderzoek naar pentesting binnen specifieke webomgevingen}

% TODO: Studentnaam en emailadres invullen
\author{Levi Van Achter}
\email{levi.vanachter@student.hogent.be}

% TODO: Medestudent
% Gaat het om een bachelorproef in samenwerking met een student in een andere
% opleiding? Geef dan de naam en emailadres hier
% \author{Yasmine Alaoui (naam opleiding)}
% \email{yasmine.alaoui@student.hogent.be}

% TODO: Geef de co-promotor op
\supervisor[Co-promotor]{S. D'Hollander (Sinergio, \href{mailto:stijn@sinergio.be}{stijn@sinergio.be})}

% Binnen welke specialisatierichting uit 3TI situeert dit onderzoek zich?
% Kies uit deze lijst:
%
% - Mobile \& Enterprise development
% - AI \& Data Engineering
% - Functional \& Business Analysis
% - System \& Network Administrator
% - Mainframe Expert
% - Als het onderzoek niet past binnen een van deze domeinen specifieer je deze
%   zelf
%
\specialisation{Mobile \& Enterprise development}
\keywords{Penetratietesten, Webapplicaties, Beveiliging, Ethical Hacking, Cybersecurity}

\begin{document}

\begin{abstract}
  Dit onderzoek richt zich specifiek op penetratietesten binnen een webomgeving, waarbij de focus ligt op het
identificeren en beveiligen van kwetsbaarheden in webapplicaties. Het omvat een grondige verkenning van
methodologieën en technieken die worden toegepast bij het testen van de beveiliging van webapplicaties. Het
onderzoek analyseert populaire webgerichte aanvalsvectoren zoals SQL-injecties, cross-site scripting (XSS) en
cross-site request forgery (CSRF). Daarnaast wordt een vergelijkende studie uitgevoerd om de effectiviteit van
de testen op meten bij 3 verschillende we applicaties.
\end{abstract}

\tableofcontents

% De hoofdtekst van het voorstel zit in een apart bestand, zodat het makkelijk
% kan opgenomen worden in de bijlagen van de bachelorproef zelf.
%---------- Inleiding ---------------------------------------------------------

% TODO: Is dit voorstel gebaseerd op een paper van Research Methods die je
% vorig jaar hebt ingediend? Heb je daarbij eventueel samengewerkt met een
% andere student?
% Zo ja, haal dan de tekst hieronder uit commentaar en pas aan.

%\paragraph{Opmerking}

% Dit voorstel is gebaseerd op het onderzoeksvoorstel dat werd geschreven in het
% kader van het vak Research Methods dat ik (vorig/dit) academiejaar heb
% uitgewerkt (met medesturent VOORNAAM NAAM als mede-auteur).
% 

\section{Inleiding}%
\label{sec:inleiding}

De groei van webapplicaties heeft de manier waarop we communiceren, winkelen en informatie delen veranderd. Deze toename in het gebruik
van webtechnologieën heeft ook nieuwe uitdagingen met zich meegebracht op het gebied van cybersecurity. Het belang van het waarborgen
van de beveiliging van webapplicaties kan niet worden overschat. Zelfs de kleinste kwetsbaarheid kan leiden tot grote gevolgen, waaronder gegevensdiefstal,
reputatieschade en financiële verliezen. Om deze reden is het uitvoeren van penetratietesten binnen een webomgeving van cruciaal
belang om potentiële beveiligingszwakheden te identificeren ~\autocite{Nagendran2019}. 

Deze bachelorproef richt zich op het verkennen
van penetratietestmethodologieën binnen een webomgeving. Ook zal er een focus gelegd worden op het begrijpen van het concept van penetratietesten,
de identificatie van veelvoorkomende beveiligingszwakheden in webapplicaties en de toepassing van praktijkgerichte onderzoeken om
de beveiliging van webapplicaties te evalueren. Het praktijkgerichte onderzoek zal worden gerealiseerd door middel van een test waarin drie
verschillende webprojecten onderworpen zullen worden aan penetratietesten. Dit onderzoek heeft als doel inzicht te krijgen in de effectiviteit van de
toegepaste methodologieën en zo vast te kunnen stellen hoe veilig verschillende soorten webprojecten zijn tegen penetratietesten.
%---------- Stand van zaken ---------------------------------------------------

\section{Literatuurstudie}%
\label{sec:literatuurstudie}

In de literatuurstudie gaan we dieper in op de
kernaspecten van penetratietesten binnen webomgevingen.
We verkennen theoretische concepten,
bestaande methodologieën en cruciale
technieken, waarbij er zowel klassieke als recente
bronnen raadplegen. Het doel is een grondig begrip
te krijgen van de huidige stand van zaken en
best practices op het gebied van webapplicatiebeveiliging.
Een essentieel startpunt is het onderzoeken
van de fundamentele principes van penetratietesten.
Hierbij wordt er gekeken naar de rol van
ethisch hacken en het identificeren van beveiligingszwakheden.
Werken zoals ”The Web Application
Hacker’s Handbook”van Dafydd Stuttard
en Marcus Pinto dienen als goede referentiepunten
~\autocite{Stuttard2011}. Vervolgens richten
we ons op het begrijpen van veelvoorkomende
webgerichte aanvalsmethoden, waaronder SQLinjecties,
cross-site scripting (XSS), cross-site request
forgery (CSRF) en andere.

Een cruciaal aspect van onze literatuurstudie
richt zich op een kritische evaluatie van diverse
webapplicatiebeveiligingstools. Het onderzoek
bevat welgekende tools zoals Burp Suite, OWASP
ZAP, Nmap... waarbij er een zo grondig mogelijk
inzicht word verkregen van hun mogelijkheden,
beperkingen en toepasbaarheid in diverse scenario's. Een specifieke bron die wordt bestudeerd,
is "An Empirical Comparison of Pen-Testing Tools
for Detecting Web App Vulnerabilities"~\autocite{Albahar2022}. Deze bron biedt onderbouwde inzichten
in de prestaties van verschillende pentesttools
bij het detecteren van kwetsbaarheden
in webapplicaties. De onderzoekers nemen de resultaten
van deze vergelijking mee in de evaluatie
om de praktische relevantie en effectiviteit van de
besproken tools beter te begrijpen.

Een opmerkelijke evolutie binnen het domein
van penetratietesten wordt gevormd door de opkomst
van geavanceerde automatiseringstools. Een
voorbeeld van zo'n innovatie is het artikel ”PentestGPT:
An LLM-empowered Automatic Penetration
Testing Tool"~\autocite{Deng2023}. Deze publicatie
schetst de opkomst van een automatisch penetratietestinstrument,
aangedreven door Large
Language Model (LLM) technologie, genaamd PentestGPT.
Dit vooruitstrevende instrument maakt
gebruik van geavanceerde algoritmes, mogelijk
gemaakt door Language Models zoals GPT (Generative
Pre-trained Transformer), om automatisch
penetratietests uit te voeren binnen webomgevingen.
Het artikel belicht hoe PentestGPT in
staat is om potentiële beveiligingszwakheden te
identificeren, kwetsbaarheden te evalueren en rapporten
te genereren, waardoor het proces van penetratietesten
grotendeels wordt gestroomlijnd.

Een essentieel element van de literatuurstudie
betreft de integratie van actuele inzichten en
best practices in de praktijk. Deze bron kijkt naar
de evolutie van automatische penetratietesten en
biedt een overzicht van de huidige stand van zaken
op dit gebied ~\autocite{AbuDabaseh2018}. De onderzoekers analyseren hoe automatische
penetratietesten zich ontwikkelen als een
veelbelovende benadering om beveiligingskwetsbaarheden
te identificeren en het testproces te
versnellen. De publicatie belicht de voordelen en
uitdagingen van geautomatiseerde tools, waarbij
de focus ligt op de integratie van AI-technologieën
en machine learning in penetratietestprocessen
% Voor literatuurverwijzingen zijn er twee belangrijke commando's:
% \autocite{KEY} => (Auteur, jaartal) Gebruik dit als de naam van de auteur
%   geen onderdeel is van de zin.
% \textcite{KEY} => Auteur (jaartal)  Gebruik dit als de auteursnaam wel een
%   functie heeft in de zin (bv. ``Uit onderzoek door Doll & Hill (1954) bleek
%   ...'')

%---------- Methodologie ------------------------------------------------------
\section{Methodologie}%
\label{sec:methodologie}

Fase 1: Identificatie van Penetratietesttools(1
week) Voer een grondige literatuurstudie uit om
relevante penetratietesttools te identificeren, waaronder
tools zoals Burp Suite, OWASP ZAP, en Nmap.
Kies tools die bekend staan om hun effectiviteit bij
het testen van webapplicaties en die geschikt zijn
voor verschillende scenario's.

Fase 2: Selectie van Webapplicaties(1 week)
Kies drie verschillende types webapplicaties om
een breed scala aan beveiligingsuitdagingen te
vertegenwoordigen. Een standaard WordPress
website zonder aanpassingen. Een voltooide Word-
Press applicatie met aangepaste functionaliteiten
en plugins. Een op Laravel gebaseerde PHPapplicatie.

Fase 3: Configuratie van Testomgeving(2 weken)
Stel voor elke webapplicatie een afzonderlijke
testomgeving in, gebruikmakend van replica's
van de live omgevingen, om realistische testresultaten
te waarborgen.

Fase 4: Voorbereiding van Penetratietesttools(1
week) Configureer elke geselecteerde tool om te
voldoen aan de specifieke kenmerken van de te
testen webapplicatie. Zorg ervoor dat de tools zijn
ingesteld om zowel geautomatiseerde als handmatige
tests uit te voeren.

Fase 5: Uitvoering van Penetratietesten(3 weken)
Voer penetratietesten uit op de drie geselecteerde
webapplicaties met behulp van de geconfigureerde
tools. Documenteer gedetailleerde resultaten,
inclusief geïdentificeerde kwetsbaarheden,
mogelijke aanvalsscenario's en beveiligingssterkten.

Fase 6: Vergelijking van Testresultaten(1 week)
Analyseer de resultaten van de penetratietesten
per applicatie en per tool. Identificeer consistent
gedetecteerde kwetsbaarheden en vergelijk de
nauwkeurigheid en diepgang van de tools in verschillende
scenario's.

Fase 7: Selectie van Bestpassende Tool(1 week)
Overweeg de bevindingen van de vergelijking en
bepaal welke tool het meest geschikt is voor welk
type webapplicatie. Houd rekening met factoren
zoals gebruiksgemak, rapportagefunctionaliteit
en snelheid van detectie.

Fase 8: Evaluatie van Geschikte Tool in Praktijktests
(2 weken) Test de geselecteerde tool opnieuw
op de drie webapplicaties om de praktische
toepasbaarheid en effectiviteit te valideren.
Documenteer eventuele verbeteringen of uitdagingen
in vergelijking met de initiële testresultaten.
%---------- Verwachte resultaten ----------------------------------------------
\section{Verwacht resultaat, conclusie}%
\label{sec:verwachte_resultaten}

De verwachte resultaten van het onderzoek
omvatten een uitgebreide vergelijking van de penetratietesttools,
waarbij zowel de sterke als zwakke
punten van elke tool worden geïdentificeerd. Deze
evaluatie richt zich specifiek op de effectiviteit van
de tools bij het detecteren van kwetsbaarheden
in diverse webapplicaties en hun vermogen om
nauwkeurige en informatieve rapporten te genereren.
Daarnaast worden de kwetsbaarheden binnen
elke webapplicatie zorgvuldig geanalyseerd. Dit
omvat een overzicht van de ernst van de geïdentificeerde
beveiligingszwakheden.

De selectie van de meest geschikte penetratietesttool
per webapplicatie wordt uitgevoerd.
Effectiviteit, nauwkeurigheid en gebruiksgemak
zijn slechts enkele van de factoren waar rekening
mee wordt gehouden. Het doel is om niet alleen
een tool te identificeren die kwetsbaarheden kan
blootleggen, maar ook om ervoor te zorgen dat
deze praktisch toepasbaar is in de specifieke context
van elke webomgeving.

Na de initiële penetratietesten volgt een fase
van herhalingstests, waarbij de geselecteerde tool
opnieuw wordt geëvalueerd in real-world scenario's.
Deze praktijkresultaten bieden inzicht in de
bruikbaarheid van de tool in dynamische omgevingen,
waarin eventuele verbeteringen of uitdagingen
worden gedocumenteerd.

Ten slotte zal het onderzoek resulteren in een
samenvattend rapport. Dit rapport omvat niet
alleen een overzicht van de bevindingen en geïdentificeerde
kwetsbaarheden, maar ook de algehele
beveiligingsstatus van elke webapplicatie.
De combinatie van deze uitgebreide resultaten
geeft waardevolle inzichten voor een doeltreffende
benadering van webapplicatiebeveiliging.
Na analyse en uitvoering van de methodologie
voor het vergelijken en testen van penetratietesttools
op diverse webapplicaties, komen verschillende
essentiële conclusies naar voren.

Allereerst biedt de vergelijking van de penetratietesttools
een inzicht in hun prestaties en
mogelijkheden. De identificatie van sterke en
zwakke punten helpt bij het vormen van een solide
besluit over welke tools het meest geschikt
zijn voor specifieke beveiligingsscenario's.
De gedetailleerde analyse van kwetsbaarheden
binnen elke webapplicatie draagt bij aan een
begrip van de specifieke beveiligingsuitdagingen
waarmee deze applicaties worden geconfronteerd.
Dit vormt de basis voor gerichte verbeteringen en
maatregelen om de algehele beveiliging te versterken.

In het samenvattend rapport worden niet alleen
de bevindingen, kwetsbaarheden en de geselecteerde
penetratietesttools gedocumenteerd,
maar ook praktische aanbevelingen voor het verbeteren
van de beveiliging. 

Deze conclusies en
aanbevelingen vormen gezamenlijk een waardevolle
bijdrage aan de ontwikkeling van effectieve
strategieën voor het waarborgen van de beveiliging
van diverse webapplicaties. Het onderzoek
biedt niet alleen inzicht in de huidige beveiligingsstatus,
maar stelt ook organisaties in staat
om proactief stappen te ondernemen ter versterking
van hun beveiligingsinfrastructuur.

\printbibliography[heading=bibintoc]

\end{document}