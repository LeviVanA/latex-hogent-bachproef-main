%%=============================================================================
%% Samenvatting
%%=============================================================================

% TODO: De "abstract" of samenvatting is een kernachtige (~ 1 blz. voor een
% thesis) synthese van het document.
%
% Een goede abstract biedt een kernachtig antwoord op volgende vragen:
%
% 1. Waarover gaat de bachelorproef?
% 2. Waarom heb je er over geschreven?
% 3. Hoe heb je het onderzoek uitgevoerd?
% 4. Wat waren de resultaten? Wat blijkt uit je onderzoek?
% 5. Wat betekenen je resultaten? Wat is de relevantie voor het werkveld?
%
% Daarom bestaat een abstract uit volgende componenten:
%
% - inleiding + kaderen thema
% - probleemstelling
% - (centrale) onderzoeksvraag
% - onderzoeksdoelstelling
% - methodologie
% - resultaten (beperk tot de belangrijkste, relevant voor de onderzoeksvraag)
% - conclusies, aanbevelingen, beperkingen
%
% LET OP! Een samenvatting is GEEN voorwoord!

%%---------- Nederlandse samenvatting -----------------------------------------
%
% TODO: Als je je bachelorproef in het Engels schrijft, moet je eerst een
% Nederlandse samenvatting invoegen. Haal daarvoor onderstaande code uit
% commentaar.
% Wie zijn bachelorproef in het Nederlands schrijft, kan dit negeren, de inhoud
% wordt niet in het document ingevoegd.

\IfLanguageName{english}{%
\selectlanguage{dutch}
\chapter*{Samenvatting}
\lipsum[1-4]
\selectlanguage{english}
}{}

%%---------- Samenvatting -----------------------------------------------------
% De samenvatting in de hoofdtaal van het document

\chapter*{\IfLanguageName{dutch}{Samenvatting}{Abstract}}

Deze bachelorproef onderzoekt de effectiviteit en gebruiksvriendelijkheid van verschillende penetratietesttools in drie specifieke webomgevingen: 
een WordPress-omgeving zonder beveiligingsplugins, een WordPress-omgeving met beveiligingsplugins, en een Laravel-applicatie. Het onderzoek richt 
zich op het belang van cybersecurity binnen de softwareontwikkeling.

De keuze voor dit onderwerp is ingegeven door de groeiende behoefte aan robuuste beveiligingsmaatregelen in softwareontwikkeling, een cruciaal 
aspect dat vaak onderbelicht blijft in de snelle evolutie van technologische innovaties. Gezien mijn achtergrond in Mobile en Enterprise 
Development, vormde de integratie van cybersecurity in ontwikkelpraktijken een natuurlijke en relevante focus voor mijn studie.

Het onderzoek werd uitgevoerd door middel van een reeks gestructureerde penetratietests, waarbij elke omgeving werd geëvalueerd op 
kwetsbaarheden met behulp van penetratietesttools met name Burp Suite, OWASP ZAP, en Metasploit. De gebruiksvriendelijkheid 
van deze tools werd ook kritisch beoordeeld, om inzicht te bieden in hoe toegankelijk deze tools zijn voor professionals in verschillende 
niveau's van hun cybersecurity-vaardigheden.

De resultaten tonen aan dat de WordPress-omgeving zonder beveiligingsplugins aanzienlijk kwetsbaarder is voor cyberaanvallen, met name 
brute force aanvallen, in vergelijking met de omgevingen waar beveiligingsplugins zijn geïmplementeerd. De Laravel-applicatie toonde 
de hoogste weerstand tegen aanvallen, wat de geavanceerde, ingebouwde beveiligingsmaatregelen van dit framework onderstreept.
Deze uitkomst lag in lijn met de verwachtingen omdat het onderzoekend deel van de studie snel duidelijk maakte hoeveel veiliger 
de security plugin wordfence in wordpress de site maakt.
Bovendien bleek uit de evaluatie dat tools zoals Burp Suite en OWASP ZAP bijzonder gebruiksvriendelijk zijn, terwijl Metasploit 
uitblinkt in zijn uitgebreide functionaliteiten voor diepgaandere penetratietests.

Deze bevindingen zijn van grote waarde voor het werkveld, omdat ze de noodzaak onderstrepen voor het integreren van effectieve 
beveiligingsstrategieën in de ontwikkelingsfase van softwareprojecten. Voor organisaties in het domein van Mobile en Enterprise 
Development suggereren de resultaten dat een investering in goede beveiligingspraktijken en tools niet alleen de veiligheid 
verbetert, maar ook de algehele kwaliteit van de ontwikkelde applicaties. Ook wijst deze studie erop hoe moeilijk het 
is om als software developer zelf pentestings uit te voeren.

Concluderend biedt dit onderzoek praktische aanbevelingen voor de selectie en implementatie van penetratietesttools, met 
nadruk op de combinatie van gebruiksvriendelijkheid en uitgebreide testcapaciteiten om een optimale beveiliging te 
garanderen. De studie benadrukt ook het belang van voortdurende educatie en aanpassing van beveiligingspraktijken aan nieuwe 
bedreigingen, wat essentieel is voor het behouden van een veilige en betrouwbare softwareontwikkelingsomgeving.
