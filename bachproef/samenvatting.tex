%%=============================================================================
%% Samenvatting
%%=============================================================================

% TODO: De "abstract" of samenvatting is een kernachtige (~ 1 blz. voor een
% thesis) synthese van het document.
%
% Een goede abstract biedt een kernachtig antwoord op volgende vragen:
%
% 1. Waarover gaat de bachelorproef?
% 2. Waarom heb je er over geschreven?
% 3. Hoe heb je het onderzoek uitgevoerd?
% 4. Wat waren de resultaten? Wat blijkt uit je onderzoek?
% 5. Wat betekenen je resultaten? Wat is de relevantie voor het werkveld?
%
% Daarom bestaat een abstract uit volgende componenten:
%
% - inleiding + kaderen thema
% - probleemstelling
% - (centrale) onderzoeksvraag
% - onderzoeksdoelstelling
% - methodologie
% - resultaten (beperk tot de belangrijkste, relevant voor de onderzoeksvraag)
% - conclusies, aanbevelingen, beperkingen
%
% LET OP! Een samenvatting is GEEN voorwoord!

%%---------- Nederlandse samenvatting -----------------------------------------
%
% TODO: Als je je bachelorproef in het Engels schrijft, moet je eerst een
% Nederlandse samenvatting invoegen. Haal daarvoor onderstaande code uit
% commentaar.
% Wie zijn bachelorproef in het Nederlands schrijft, kan dit negeren, de inhoud
% wordt niet in het document ingevoegd.

\IfLanguageName{english}{%
\selectlanguage{dutch}
\chapter*{Samenvatting}
\lipsum[1-4]
\selectlanguage{english}
}{}

%%---------- Samenvatting -----------------------------------------------------
% De samenvatting in de hoofdtaal van het document

\chapter*{\IfLanguageName{dutch}{Samenvatting}{Abstract}}

Deze bachelorproef voert onderzoek naar het belang van security binnen softwareontwikkeling en analyseert de effectiviteit en 
gebruiksvriendelijkheid van verschillende penetratietesttools in drie specifieke webomgevingen, met name  
een WordPress-omgeving zonder beveiligingsplugins, een WordPress-omgeving met beveiligingsplugins, en een Laravel-applicatie.

De keuze voor dit onderwerp werd vooral ingegeven door de groeiende behoefte aan robuuste beveiligingsmaatregelen in softwareontwikkeling, een cruciaal 
aspect dat vaak onderbelicht blijft in de snelle evolutie van technologische innovaties. 

Het onderzoek werd uitgevoerd door middel van een reeks gestructureerde penetratietests op verschillende omgevingen, waarbij een evaluatie op 
kwetsbaarheden werden gemaakt met behulp van drie tools Burp Suite, OWASP ZAP en Metasploit. Naast de effectiviteit werd ook op gebruiksvriendelijkheid en andere factoren 
ervan kritisch beoordeeld. Zo tracht dit onderzoek inzicht te bieden hoe toegankelijk deze tools zijn voor professionals in verschillende 
niveaus van cybersecurity.

De resultaten van de analyses tonen aan dat een WordPress-omgeving zonder beveiligingsplugins aanzienlijk kwetsbaarder is voor cyberaanvallen, in het bijzonder  
brute force- en SQL-injectie aanvallen in vergelijking met omgevingen waar beveiligingsplugins zijn geïmplementeerd. De Laravel- 
en Wordpress-applicatie met een beveiligingsplugin toonde 
de hoogste weerstand tegen aanvallen, wat de geavanceerde, ingebouwde beveiligingsmaatregelen van deze frameworks benadrukt.
Deze uitkomst ligt in lijn met de verwachtingen omdat het onderzoekend deel van de studie snel duidelijk maakte hoeveel veiliger 
de security plugin wordfence in Wordpress de site maakt.

Daarnaast bleek uit de evaluatie dat tools zoals Burp Suite en OWASP ZAP bijzonder gebruiksvriendelijk zijn, terwijl Metasploit 
uitblinkt in zijn verbruiks-efficiëntie tijdens het uitvoeren van een pentest. Met meer beschikbare tijd en middelen zou ook 
de Professional versie van Burp Suite getest kunnen zijn, wat relevant zou zijn aangezien er dan gemeten zou kunnen worden of deze 
laatste met alle beschikbare betalende functionaliteiten een groot verschil zou vormen ten opzichte van de gratis versie.

Deze bevindingen zijn van grote waarde voor het werkveld, omdat ze de noodzaak onderstrepen voor het integreren van effectieve 
beveiligingsstrategieën in de ontwikkelingsfase van softwareprojecten. Voor organisaties in het domein van Mobile en Enterprise 
Development suggereren de resultaten dat een investering in goede beveiligingspraktijken en tools niet alleen de veiligheid 
verbetert, maar ook de algehele kwaliteit van de ontwikkelde applicaties. Tevens geeft deze studie de complexiteit weer 
om als software developer zelf pentestings uit te voeren.

Tot slot biedt dit onderzoek praktische aanbevelingen voor de selectie en implementatie van penetratietesttools, 
om een optimale beveiliging te 
garanderen. De studie benadrukt ook het belang van voortdurende educatie en aanpassing van beveiligingspraktijken aan nieuwe 
bedreigingen.
