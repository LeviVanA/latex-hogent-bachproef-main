%%=============================================================================
%% Conclusie
%%=============================================================================

\chapter{Conclusie}%
\label{ch:conclusie}

% TODO: Trek een duidelijke conclusie, in de vorm van een antwoord op de
% onderzoeksvra(a)g(en). Wat was jouw bijdrage aan het onderzoeksdomein en
% hoe biedt dit meerwaarde aan het vakgebied/doelgroep? 
% Reflecteer kritisch over het resultaat. In Engelse teksten wordt deze sectie
% ``Discussion'' genoemd. Had je deze uitkomst verwacht? Zijn er zaken die nog
% niet duidelijk zijn?
% Heeft het onderzoek geleid tot nieuwe vragen die uitnodigen tot verder 
%onderzoek?

Op basis van de resultaten in hoofdstuk 4 van deze studie kunnen we de initiële onderzoeksvragen als volgt beantwoorden:

\begin{enumerate}
  \item 	Hoe variëren de prestaties van verschillende penetratietesttools bij het identificeren van kwetsbaarheden binnen de drie specifieke webomgevingen.
  
  De prestaties van de penetratietesttools Burp Suite, Metasploit en OWASP ZAP varieerden significant bij het identificeren 
  van kwetsbaarheden binnen de drie specifieke webomgevingen.
  \begin{itemize}
    \item Bij het uitvoeren van een brute force-aanval op een WordPress-omgeving beveiligd met de Wordfence-plugin, 
    lukte het Burp Suite om niet om het wachtwoord te kraken aangezien er een max van 10 pogingen was. De Community Edition 
    van Burp Suite was echter beperkt in functionaliteit, wat de volledigheid van de penetratietest beperkte. De 
    gebruiksvriendelijkheid van Burp Suite was redelijk, met een gebruikersscore van 7/10, vooral dankzij de uitgebreide 
    documentatie en tutorials beschikbaar online.
    \item Metasploit, bediend via de terminal, slaagde er eveneens niet in om het wachtwoord te onthullen, dit is wederom het 
    gevolg van een maximum aantal inlogpogingen. De gebruiksvriendelijkheid van Metasploit werd beoordeeld 
    met een 6/10, vanwege de noodzaak om command-line vaardigheden te bezitten, wat veel technische kennis vergt. 
    De uitgebreide scala aan modules maakt Metasploit anderzijds een waardevolle keuze voor professionele penetratietesters.
    \item OWASP ZAP leverde vergelijkbare resultaten als de betaalde versie van Burp Suite, kon ook het wordpress-framework 
    niet kraken door het beperkt aantal beschikbare inlogpogingen. De gebruiksvriendelijkheid van OWASP ZAP was hoog, met een score van 8/10, 
    dankzij de intuïtieve interface. Het feit dat het een open-source tool is maakt het een aantrekkelijke keuze voor 
    organisaties met beperkte budgetten voor beveiligingstests.
  \end{itemize}
  Ook bij de SQL-injectie pentest bleek de beveiliging robuust, met effectieve detectie en blokkering van de aanvalspogingen.
  Dit toont aan dat de wordfence plugin een effectieve beveiligingslaag vormt tegen bekende kwetsbaarheden.
  
  \item Welke specifieke kwetsbaarheden detecteren de penetratietesttools in een WordPress-omgeving zonder beveiligingsplugins en hoe verschilt dit van de omgevingen met beveiligingsplugins en de Laravel-applicatie?
  
  in een WordPress-omgeving zonder beveiligingsplugins werden meerdere kwetsbaarheden blootgelegd door de penetratietesttools:
  \begin{itemize}
    \item De penetratietesttools identificeerden het wachtwoord in 90\% van de gevallen binnen de 3 minuten. Dit benadrukt de 
    hoge kwetsbaarheid van onbeveiligde WordPress-sites. Dit was te danken aan de zwakke wachtwoorden en gebruikersnamen en 
    de mogelijkheid om een onbeperkt aantal inlogpogingen uit te voeren.
    \item De beveiligingsplugins, zoals Wordfence, reduceerden het aantal succesvolle brute force-aanvallen drastisch. 
    De tools konden slechts in 10\% van de gevallen het wachtwoord kraken, dankzij inloglimieten en 
    real-time alerts die door de plugins werden geactiveerd.
    \item De Laravel-applicatie toonde robuuste beveiligingsprestaties. In geen van de gevallen lukte het om het wachtwoord 
    te kraken. De ingebouwde beveiligingsmaatregelen, zoals geavanceerde gebruikersauthenticatie en 
    versleutelingstechnieken, boden een solide bescherming.
  \end{itemize}
  \item In hoeverre verbeteren beveiligingsplugins de detectiecapaciteiten van deze tools op een WordPress-site?
  \begin{itemize}
    \item Beveiligingsplugins beperkten het aantal mislukte inlogpogingen tot maximaal 20 binnen een periode van 
    30 minuten. Na het overschrijden van deze limiet, werd de account tijdelijk vergrendeld en werd een alert 
    verzonden met details zoals IP-adres, hostname, en locatie van de aanvaller.
    \item Wordfence stuurde real-time alerts wanneer verdachte activiteiten werden gedetecteerd. Tijdens de 
    tests werd elke brute force-aanval binnen 2 minuten gedetecteerd en geblokkeerd.
    \item OWASP ZAP leverde vergelijkbare resultaten als de betaalde versie van Burp Suite, met een detectiegraad van 80\% 
    voor bekende kwetsbaarheden binnen 6 minuten. De gebruiksvriendelijkheid en open-source aard van OWASP ZAP maken het een 
    aantrekkelijke optie voor organisaties met beperkte budgetten voor beveiligingstests.
  \end{itemize}
\end{enumerate}

Samenvattend, het onderzoek toont aan dat beveiligingsplugins een cruciale rol spelen in het versterken van de 
beveiliging van WordPress-omgevingen, waardoor de detectie en blokkering van aanvallen veel verbeteren. 
De keuze van penetratietesttools moet afgestemd zijn op de specifieke behoeften en de complexiteit van de te 
testen omgeving. Laravel-applicaties profiteren van hun robuuste ingebouwde beveiligingsarchitectuur, wat het 
merkbaar veiliger maakt in vergelijking met standaard WordPress-omgevingen. De gebruiksvriendelijkheid van de 
tools speelt ook een belangrijke rol, waarbij OWASP ZAP opvalt als een intuïtieve en kosteneffectieve optie 
voor beveiligingstests
