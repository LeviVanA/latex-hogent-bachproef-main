%%=============================================================================
%% Conclusie
%%=============================================================================

\chapter{Conclusie}%
\label{ch:conclusie}

% TODO: Trek een duidelijke conclusie, in de vorm van een antwoord op de
% onderzoeksvra(a)g(en). Wat was jouw bijdrage aan het onderzoeksdomein en
% hoe biedt dit meerwaarde aan het vakgebied/doelgroep? 
% Reflecteer kritisch over het resultaat. In Engelse teksten wordt deze sectie
% ``Discussion'' genoemd. Had je deze uitkomst verwacht? Zijn er zaken die nog
% niet duidelijk zijn?
% Heeft het onderzoek geleid tot nieuwe vragen die uitnodigen tot verder 
%onderzoek?

Dit onderzoek heeft de effectiviteit en gebruiksvriendelijkheid van verschillende penetratietesttools onderzocht in het 
detecteren van kwetsbaarheden binnen drie specifieke webomgevingen: een WordPress-omgeving zonder beveiligingsplugins, 
een WordPress-omgeving met beveiligingsplugins, en een Laravel-applicatie. De onderzoeksvragen waren gericht op het 
analyseren van de prestaties van deze tools in elk van deze omgevingen, het onderzoeken van de impact van beveiligingsplugins 
op de detectiecapaciteiten binnen WordPress, en het vergelijken van de algemene beveiligingsniveaus tussen een onbeveiligde 
WordPress-website, een beveiligde site en een Laravel-applicatie.

De resultaten uit de brute force tests toonden significante verschillen in veiligheid tussen de geteste omgevingen. 
De WordPress-omgeving zonder beveiligingsplugins was opvallend kwetsbaar, met meerdere succesvolle inlogpogingen door 
brute force aanvallen, voornamelijk door het ontbreken van beperkingen op het aantal inlogpogingen. Dit onderstreept 
de dringende behoefte aan robuuste beveiligingsmaatregelen in dergelijke basisconfiguraties om de integriteit van 
systemen te waarborgen.

Daarentegen vertoonde de WordPress-omgeving met beveiligingsplugins een aanzienlijk hogere weerstand tegen deze 
aanvallen. De beveiligingsplugins waren effectief in het verminderen van de kwetsbaarheid door het blokkeren van 
herhaaldelijke inlogpogingen en het verstrekken van real-time waarschuwingen, wat aantoont dat deze maatregelen 
essentieel zijn voor het verhogen van de weerbaarheid tegen externe bedreigingen.

De Laravel-applicatie toonde de hoogste weerstand tegen de brute force aanvallen, wat getuigt van de robuuste, 
ingebouwde beveiligingsfuncties van het framework. Laravel's uitgebreide authenticatiemethoden, waaronder ingebouwde 
hash-algoritmen en de mogelijkheid tot eenvoudige integratie van tweefactorauthenticatie, evenals het gebruik van 
middleware voor beveiligingscontroles, bieden een sterke verdediging die effectief ongeautoriseerde toegangspogingen 
afweert. Deze geïntegreerde benadering van beveiliging maakt Laravel bijzonder geschikt voor bedrijven die complexe 
applicaties met gevoelige data beheren en hoge beveiligingseisen stellen.

De evaluatie van de gebruiksvriendelijkheid van de tools onthulde dat Burp Suite en OWASP ZAP uitblinken in termen 
van eenvoud en efficiëntie, waardoor ze ideaal zijn voor teams die snel en effectief kwetsbaarheden willen identificeren 
en aanpakken. Metasploit onderscheidde zich door zijn uitgebreide reeks pentesting-functionaliteiten, wat het een 
krachtige tool maakt voor meer complexe beveiligingsbehoeften en diepgaande testscenario's.

Deze bevindingen dragen bij aan het vakgebied door het belang van het implementeren van effectieve beveiligingsmaatregelen 
in WordPress en het potentieel van frameworks zoals Laravel voor het bouwen van veiligere webapplicaties te benadrukken.
 Verder onderstrepen ze het voordeel van het kiezen van een framework dat niet alleen veiliger is door zijn architectuur,
  maar ook ontwikkelaars aanmoedigt om vanaf het begin beveiliging als een prioriteit te beschouwen. Deze inzichten bieden 
  waardevolle informatie voor ontwikkelaars en beveiligingsexperts bij het maken van geïnformeerde keuzes bij het 
  selecteren van technologieën voor hun projecten.

Een belangrijke conclusie van dit onderzoek is dat het onverstandig is om niet op slechts één enkele pentesting tool te 
vertrouwen. Het gebruik van Nmap in combinatie met Burp Suite toont aan hoe belangrijk het is om verschillende tools te 
integreren voor een meer omvattende beveiligingstest. Verder onderzoek en diepgaandere analyses zijn 
nodig om de onderliggende mechanismen van deze tools beter te begrijpen en om hun langetermijneffecten in diverse 
aanvalsscenario's te evalueren, wat zal leiden tot de ontwikkeling van robuustere en veiligere webomgevingen.