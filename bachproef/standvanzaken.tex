\chapter{\IfLanguageName{dutch}{Stand van zaken}{State of the art}}%
\label{ch:stand-van-zaken}

% Tip: Begin elk hoofdstuk met een paragraaf inleiding die beschrijft hoe
% dit hoofdstuk past binnen het geheel van de bachelorproef. Geef in het
% bijzonder aan wat de link is met het vorige en volgende hoofdstuk.

% Pas na deze inleidende paragraaf komt de eerste sectiehoofding.
\section{\IfLanguageName{dutch}{doel van het onderzoek}{goal of the study}}

Het hoofddoel van dit onderzoek is het grondig evalueren van de effectiviteit van verschillende penetratietesttools in twee typen webomgevingen: 
WordPress sites voorzien van beveiligingsplugins, een wordpress website zonder aanpassingen en een op maat gemaakte Laravel applicatie. Deze evaluatie 
heeft als doel de sterke en zwakke punten van elke pentest tool te identificeren in hun vermogen om kwetsbaarheden te detecteren en weerstand te bieden 
tegen cyberdreigingen. Daarnaast wordt ook de invloed van de structuur van de webomgeving op de prestaties van deze tools onderzocht.
Dit onderzoek maakt gebruik van inzichten uit een recente publicatie in MDPI Electronics, die een uitgebreide analyse van cybersecuritydreigingen en 
de effectiviteit van verschillende verdedigingsmechanismen binnen webapplicaties biedt. Het doel is om de kennis over hoe webapplicaties beter 
beschermd kunnen worden tegen geavanceerde cyberaanvallen te verdiepen. Dit artikel benadrukt specifiek het belang van voortdurende vernieuwing 
in beveiligingsstrategieën om voldoende in te kunnen spelen op de snel evoluerende cyberdreigingen, wat direct gepaard gaat met de doelstellingen van dit 
onderzoek ~\autocite{Altulaihan2023}.

\section{\IfLanguageName{dutch}{Pentesting}{Pentesting}}
\label{sec:pentesting}
Penetration testing (pentesting) is een cruciaal onderdeel bij het waarborgen van netwerk-, systeem- en appllicatiebeveiliging. 
Wetenschappelijke studies tonen aan dat het proces van pentesting verschillende methoden omvat, zoals white box, 
black box, en grey box, elk met hun eigen aanpak van kwetsbaarheden in een systeem. Het doel van pentesting is om 
veiligheidszwakheden te identificeren en te exploiteren\footnote{het proces waarbij een tester actief gebruikmaakt 
van veiligheidszwakheden of kwetsbaarheden in een systeem om te laten zien hoe een kwaadwillende aanvaller deze kan misbruiken.} 
om organisaties te helpen begrijpen waar hun beveiligingskwestbaarheden zich bevinden ~\autocite{Alhamed2023}.

Een penetratietest bestaat uit drie belangrijke fasen: voorbereiding, uitvoering en analyse. Gedurende deze fasen gebruiken 
testers diverse tools zoals Nmap, Metasploit en Nessus. Tijdens de voorbereiding worden de scope en doelstellingen vastgesteld. 
Vervolgens vindt de test plaats, waarbij tools en methoden worden toegepast om kwetsbaarheden te identificeren. 
Ten slotte worden de resultaten geanalyseerd en gerapporteerd, inclusief aanbevelingen voor het verbeteren van de beveiliging 
~\autocite{Sarker2023}.

Bij het selecteren van tools en methoden voor penetratietests moet rekening worden gehouden met verschillende factoren, zoals de 
grootte van het netwerk, het soort infrastructuur en het type kwetsbaarheden dat wordt getest. Wetenschappelijke studies 
benadrukken het belang van een grondige planning en voorbereiding om ervoor te zorgen dat de test nauwkeurig en doeltreffend 
wordt uitgevoerd ~\autocite{Alhamed2023}.

Een andere overweging bij penetratietests is het volgen van specifieke normen en richtlijnen, zoals ISO 27000, die ethische 
aspecten en best practices voor pentesting benadrukken ~\autocite{DalalanaBertoglio2017}. Recente ontwikkelingen in technologie, zoals deep 
reinforcement learning\footnote{Bij deep reinforcement learning combineren we de patroonherkenning van deep learning en neurale 
netwerken met het feedback-gebaseerde leren van versterkend leren. Zo kunnen computers intelligente beslissingen nemen en complexe 
taken uitvoeren door te leren van hun interacties met de omgeving}, bieden geavanceerde benaderingen voor automatische pentesting. 
Deze nieuwe methoden kunnen de efficiëntie van het proces verhogen en bijdragen aan een betere detectie van kwetsbaarheden 
~\autocite{Yi2023}.

Voor gedetailleerde informatie en diepgaande analyses van penetratietests in verschillende contexten kunnen wetenschappelijke 
artikelen en studies worden geraadpleegd, die het onderwerp uitvoerig behandelen ~\autocite{Sarker2023}.

\section{\IfLanguageName{dutch}{Webomgevingen}{Web environments}}
\label{sec:Webomgevingen}

\subsection{\IfLanguageName{dutch}{Veiligheidskwetsbaarheden in Webapplicaties}{Security vulnerabilities in Web applications}}
\label{sec:Veiligheidskwetsbaarheden in Webomgevingen}
\begin{figure}
    \centering
    \includegraphics[height=0.3\textheight]{wordfence_stopped_attacks.png}
    \caption[Totaal aantal geblokkeerde aanvallen: Wordfence netwerk]{Totaal aantal geblokkeerde aanvallen: Wordfence netwerk}
\end{figure}

In de methodologie bij het testen van webomgevingen met name WordPress en Laravel, speelt het identificeren van veelvoorkomende 
kwetsbaarheden een belangrijke rol. Zoals besproken in vorig hoofdstuk valt het onderzoeken van kwetsbaarheden in webapplicaties onder 
de fase voorbereiding van een pentest.

In WordPress zonder beveiligingsplugins moeten misconfiguraties zoals onbeschermde back-ups 
en tijdelijke bestanden worden geïdentificeerd, omdat ze gevoelige informatie kunnen lekken. Deze onbeschermde bestanden kunnen 
ertoe leiden dat een aanvaller toegang krijgt tot gevoelige informatie, zoals database-informatie in het geval van 
wp-config.php-bestanden\footnote{een bestand dat over zeer belangrijke informatie over een website beschikt} 
~\autocite{DalalanaBertoglio2017}.

Bij het testen van WordPress met beveiligingsplugin(s) is het belangrijk om de effectiviteit van beveiligingsmaatregelen te 
evalueren. Op bovenstaande foto kan je zo zien hoeveel aanvallen het wordfence netwerk per dag tegenhoud. Kwetsbaarheden 
zoals SQL-injecties en Cross-Site Scripting (XSS) zijn belangrijke aandachtsgebieden voor pentesters, vooral in WordPress-omgevingen 
met beveiligingsplugin(s) ~\autocite{Albahar2022}.

Bij Laravel-applicaties is het belangrijk om te letten op backend-logica, authenticatie en sessiebeheer. 
Mass assignment-kwetsbaarheden moeten zorgvuldig worden onderzocht om ongeautoriseerde gegevenswijzigingen te voorkomen.
Mass assignment is een techniek binne Eloquent\footnote{ORM van Laravel} waarbij een gebruiker meerdere velden tegelijk kan bijwerken, wat een beveiligingsrisico kan vormen. 
Daarnaast is het essentieel om te testen op SQL-injecties en te zorgen voor veilige authenticatie in Laravel-applicaties 
~\autocite{Altulaihan2023}.

Een effectieve methodologie omvat een grondige beoordeling van de infrastructuur, identificatie van kwetsbaarheden en het 
testen van beveiligingsmaatregelen met diverse tools zoals Metasploit, Burp Suite en OWASP ZAP ~\autocite{Ravindran2022}. 
Het uiteindelijke doel is om kwetsbaarheden te ontdekken en effectieve maatregelen te nemen om de beveiliging van webomgevingen 
te versterken.

\begin{figure}
    \centering
    \includegraphics[height=0.3\textheight]{data-breaches-statistics-us.png}
    \caption[Aantal data breaches in de US ]{Aantal data breaches in de US }
\end{figure}

\subsection{\IfLanguageName{dutch}{Security mechanismen in Webapplicaties}{Security mechanisms in Web applications}}
\label{sec:Security mechanismen in Webomgevingen}

In dit hoofdstuk worden de security mechanismen van drie verschillende webomgevingen onderzocht: WordPress zonder beveiligingsplugins, 
WordPress met de Wordfence plugin, en het Laravel framework. Elk platform heeft zijn eigen beveiligingsmaatregelen die bijdragen 
aan de algehele veiligheid van webapplicaties.

WordPress, een veelgebruikt content management systeem, bevat enkele ingebouwde beveiligingsfuncties zoals gebruikersrollen, beveiligde 
wachtwoorden en bestandsrechten. Hoewel deze maatregelen helpen bij het beschermen van een website, vooral tegen basisaanvallen, zijn ze 
mogelijk niet afdoende tegen geavanceerde bedreigingen zoals SQL-injecties en cross-site scripting (XSS). Het gebruik van aanvullende 
beveiligingsmaatregelen, zoals beveiligingsplugins, wordt daarom aanbevolen, vooral voor websites die gevoelige informatie verwerken 
of een hoog risico lopen op aanvallen~\autocite{Trunde2015}

Wordfence, een vooraanstaande beveiligingsplugin voor WordPress, biedt een uitgebreide suite van beveiligingsfuncties, waaronder een 
firewall, malware scanner en DDoS bescherming. Deze plugin is speciaal ontworpen om WordPress-websites te beschermen tegen bekende 
bedreigingen en om verdachte activiteiten in realtime te monitoren. De effectiviteit van Wordfence in het verdedigen van WordPress-sites 
tegen een breed scala aan aanvallen is erkend en heeft bijgedragen aan zijn reputatie als een betrouwbare beveiligingsoplossing voor 
WordPress-gebruikers~\autocite{277144}.

Laravel, een krachtig PHP-framework, geniet bekendheid vanwege zijn geavanceerde beveiligingsfuncties en solide architectuur. Het 
framework biedt ingebouwde bescherming tegen veelvoorkomende beveiligingskwetsbaarheden, zoals CSRF-aanvallen en SQL-injecties, wat 
bijdraagt aan de veiligheid van webapplicaties die op Laravel zijn gebouwd. Bovendien biedt Laravel's uitgebreide authenticatie- en 
autorisatiesysteem ontwikkelaars een gestandaardiseerde en veilige methode om gebruikers te beheren en toegangscontroles toe te passen. 
Deze functies maken Laravel een populaire keuze voor ontwikkelaars die streven naar veilige en betrouwbare webapplicaties
~\autocite{Adamu2020}.

Het toevoegen van beveiligingsplugins zoals Wordfence aan WordPress kan de beveiliging aanzienlijk verbeteren, terwijl Laravel van nature een 
sterke beveiligingslaag biedt. Het kiezen van het juiste platform hangt af van de specifieke behoeften van het project en de nadruk op beveiliging.
\section{\IfLanguageName{dutch}{Tools}{Tools}}
\label{sec:Webomgevingen}
Bij het kiezen van de juiste pentesting tools voor webomgevingen moet, worden overwogen welke functionaliteiten 
elke tool biedt. Dit is van cruciaal belang, gezien de vele verschillende soorten webapplicaties en de bijbehorende veiligheidsrisico's.

Een van de eerste aspecten waarop moet worden gelet, is of de tool geschikt is voor de specifieke behoeften. Sommige tools 
zijn specifiek ontworpen voor het testen van websites, terwijl andere meer geschikt zijn voor het evalueren van volledige 
netwerken. Het is essentieel om een tool te selecteren die aansluit bij de specifieke vereisten van de situatie. Het 
is logisch om voor verschillende taken verschillende tools te gebruiken~\autocite{Deepikakongara2023}.
Zoals aangehaalt in het eerste hoofdstuk, is het belangrijk om rekening te houden met de grootte van het netwerk, 
het soort infrastructuur en het type kwetsbaarheden dat wordt getest.

Gebruiksgemak is ook van belang, vooral wanneer niet alle teamleden experts zijn. Een tool zoals OWASP ZAP, 
die gebruiksvriendelijk is, kan aanzienlijk veel tijd besparen en de toegankelijkheid voor alle teamleden vergroten.
Daarnaast biedt een analyse, toegankelijk via Google Books, een volledige vergelijking tussen enkele van de meest gekende pentesting tools op de markt, 
zoals Nmap, Burp Suite, en OWASP ZAP. Deze bron is bijzonder waardevol voor professionals binnen de cybersecurity wereld, omdat het inzicht geeft in de sterke en zwakke 
punten van elk van deze tools. Door deze tools te vergelijken, kunnen teams betere beslissingen nemen over welke tools het meest geschikt zijn 
voor hun specifieke beveiligingsbehoeften ~\autocite{Velu2022}.

Regelmatige updates zijn van groot belang om ervoor te zorgen dat de tool in staat blijft om nieuwe kwetsbaarheden te 
identificeren. Het is belangrijk om niet achter te blijven en verouderde bedreigingen te testen terwijl kwaadwillende 
partijen al nieuwe methoden hebben ontwikkeld.

Natuurlijk speelt budgettaire overweging ook een rol, aangezien tools variëren van gratis tot zeer prijzig. Het 
is van belang om een tool te vinden die binnen het budget past maar tegelijkertijd voldoet aan de vereiste functionaliteiten.

Ten slotte moet de tool voldoende diepgaand zijn om zowel oppervlakkige als meer verborgen problemen aan te pakken. 
Een goede tool kan helpen bij het identificeren en verhelpen van alles, van kleine fouten tot ernstige beveiligingslekken
~\autocite{Maji2022}.

Door deze aspecten af te wegen, kunnen de juiste tools worden geselecteerd voor pentestingwerkzaamheden. 
Dit kan veel bijdragen aan het verbeteren van de veiligheid van webomgevingen en het uitvoeren van effectievere beveiligingstests.

\section{\IfLanguageName{dutch}{wat weten we uit de literatuur}{what is known from the literature}}
\label{sec:wat-weten-we-uit-de-literatuur}
Beveiliging van webapplicaties is een noodzakelijk onderwerp geworden. Recente onderzoeken en literatuur leveren 
inzichten in verschillende aspecten van dit vakgebied, waaronder de effectiviteit van penetratietests, het belang van beveiligingsplugins voor WordPress en de verschillen 
in beveiligingsmaatregelen tussen webapplicatieframeworks zoals Laravel en WordPress. Een overzicht van wat reeds gekend is op dit gebied, ondersteund door een aantal 
bronnen, toont de huidige kennis en praktijken aan.

Cybersecurity-instrumenten spelen een cruciale rol, bijvoorbeeld bij penetratietests binnen webomgevingen. Het weloverwogen 
selecteren van effectieve tools is essentieel, zoals ik in vorig hoofdstuk heb aangehaalt, om kwetsbaarheden in webapplicaties te identificeren en aan te pakken. 
Deze selectie vereist niet alleen technische kennis, maar ook strategisch inzicht in hoe verschillende tools verschillende 
soorten beveiligingslekken kunnen identificeren.
Een doordachte aanpak bij het kiezen van penetratietesttools versterkt niet alleen de beveiligingshouding 
van organisaties, maar verbetert ook hun vermogen om zich aan te passen aan nieuwe bedreigingen. Het consequent toepassen 
van deze tools helpt niet alleen bij het identificeren van onmiddellijke dreigingen, maar biedt ook inzichten in potentiële 
toekomstige kwetsbaarheden ~\autocite{Albahar2022}.
Daarnaast draagt een effectieve beveiligingsstrategie bij aan het opbouwen van vertrouwen bij klanten en gebruikers, die er 
zeker van kunnen zijn dat hun gegevens veilig worden beheerd. In het licht van toenemende regelgeving rond gegevensbescherming 
is het ook van groot belang dat organisaties niet alleen voldoen aan de industrienormen, maar deze zelfs overtreffen.
Dus, terwijl de technologie blijft evolueren, is het van cruciaal belang dat beveiligingsteams hun tools voortdurend 
beoordelen en bijwerken, zodat ze voorbereid zijn op zowel de huidige als toekomstige cyberuitdagingen. Dit continue 
proces van beoordeling en verbetering is essentieel voor het bewaren van een sterke verdediging tegen een breed scala 
aan internetbedreigingen.

Een belangrijk aspect van onderzoek draait om de impact van beveiligingsplugins op WordPress-websites. Deze plugins, zoals Wordfence en iThemes Security, worden 
erkend vanwege hun vermogen om een cruciale beveiligingslaag toe te voegen aan WordPress-sites. Ze zijn speciaal ontworpen om een vele bedreigingen af 
te weren, waardoor ze een belangerijk onderdeel vormen van de beveiligingsstrategie voor elke WordPress-omgeving~\autocite{Casola2020}.

Een rapport dat werd gepresenteerd, onderzoekt de drie belangrijkste beveiligingsrisico's geïdentificeerd door OWASP en verkent hoe penetratietesttools kunnen 
worden ingezet om deze bedreigingen bij websites op te sporen. Deze analyse is van groot belang voor organisaties die streven naar het verbeteren van hun 
weerbaarheid tegen veelvoorkomende cyberaanvallen ~\autocite{Sharma2023}.

Ten slotte valt op dat Laravel aanzienlijke beveiligingsvoordelen biedt in vergelijking met WordPress. Deze voordelen worden voornamelijk toegeschreven aan 
de robuuste architectuur van Laravel, die inherent meer beveiligingslagen biedt. Laravel maakt gebruik van moderne beveiligingspraktijken en heeft ingebouwde 
functies die helpen bij het beschermen tegen veelvoorkomende bedreigingen zoals SQL-injecties, cross-site scripting (XSS) en cross-site request forgery (CSRF)~\autocite{Lebedeva2023}. Deze focus 
op beveiliging maakt Laravel een aantrekkelijke keuze voor ontwikkelaars en bedrijven die de veiligheid van hun webapplicaties serieus nemen en zich bewust 
zijn van de toenemende dreigingen op het internet. De voordelen van Laravel op het gebied van beveiliging zijn dus een belangrijke overweging voor organisaties 
bij het selecteren van een framework voor hun projecten.

Samenvattend brengen deze studies de kennis die beschikbaar is over de beveiliging van webapplicaties aan licht. 
Het is essentieel om inzicht te krijgen in de specifieke beveiligingsproblemen die elk webapplicatieframework met zich meebrengt. Door deze inzichten 
toe te passen kunnen organisaties hun beveiligingshouding versterken en beter beschermen tegen de voortdurende dreiging van cyberaanvallen.

\section{\IfLanguageName{dutch}{wat weten we nog niet uit de literatuur}{what is not known from the literature}}
Hoewel de bestaande literatuur een waardevolle kijk biedt op het vlak van webapplicatiebeveiliging en de effectiviteit van penetratietesting, blijven er nog 
zaken die niet volledig zijn onderzocht. Uit de analyse van twee specifieke bronnen, de conferentieproceedings van CyberCon en een publicatie in het 
Computer Science and Information Technology (CS and IT) tijdschrift, komen enkele van deze tekorten aan bod.

Een artikel gepresenteerd op de CyberCon-conferentie benadrukt de complexiteit van beveiligingsbedreigingen en stelt vragen bij het ruime assortiment 
van bestaande pentesting tools en -methoden om deze bedreigingen effectief aan te pakken. Het artikel zegt dat, ondanks de brede toepassing en 
ontwikkeling van deze tools, er een diepgaander begrip nodig is. Het is cruciaal om te begrijpen hoe nieuwe soorten cyberaanvallen, vooral waarbij 
geavanceerde kunstmatige intelligentie en machine learning technieken gebruiken, effectief geïdentificeerd kunnen worden.~\autocite{Petrica2022}.

Daarnaast wijst een studie gepubliceerd in het CS and IT tijdschrift op het ontbreken van benchmarks en vergelijkende studies die de prestaties van verschillende 
beveiligingsplugins en frameworks onder bepaalde omstandigheden beoordelen. Dit gebrek aan vergelijkend onderzoek maakt het voor ontwikkelaars 
moeilijk om onderbouwde beslissingen te nemen bij de implementatie van beveiligingsmaatregelen binnen hun webapplicaties. De studie benadrukt de noodzaak voor 
het maken van benchmarks, die van belang zijn bij het vinden van de meest effectieve beveiligingsaanpakken tegen 
diverse cyberdreigingen ~\autocite{AbuDabaseh2018}.

Beide bronnen tonen het belang van voortdurend onderzoek naar en ontwikkeling van penetratietesting methodologieën, beveiligingsplugins en frameworks om te kunnen 
blijven voldoen aan de eisen van een voortdurend veranderende omgeving. Ze benadrukken dat, hoewel veel bekend is over de basisprincipes van webapplicatiebeveiliging, 
de details van het beschermen tegen de nieuwste en meest geavanceerde aanvalstechnieken nog ontdekt moeten worden. Dit duidt op een enorme behoefte aan 
specifiek onderzoek en praktijkexperimenten om de effectiviteit van bestaande en nieuwe beveiligingsmethoden te verbeteren.

\section{\IfLanguageName{dutch}{relevantie van het onderzoek}{relevance of the study}}
\subsection{\IfLanguageName{dutch}{theoretische relevantie}{theoretical relevance}}
De inzichten uit de Pertanika Journals ~\autocite{Jarupunphol2023}, die een bijdrage levert aan het verbreden van de theoretische basis van webapplicatiebeveiliging.
Dit onderzoek, zoals besproken in de Pertanika Journals, vergelijkt verschillende penetratietesttools binnen een specifieke webomgeving. Het biedt 
een solide basis voor een diepgaand begrip van hoe deze tools verschillende kwetsbaarheden benaderen en analyseren. Dit draagt bij aan de ontwikkeling 
van een theoretisch kader dat de verschillende pentesting tools naast elkaar plaatst.

De verkregen inzichten dragen bij aan een breder begrip van webapplicatiebeveiliging. Het vergelijken van verschillende penetratietesttools binnen een specifieke 
webomgeving biedt waardevolle inzichten in hoe deze tools kwetsbaarheden benaderen en analyseren. Dit draagt bij aan het ontwikkelen van een algeheel begrip van 
pentesting tools, wat van cruciaal belang is voor professionals in de cybersecurity en webapplicatieontwikkeling om effectieve beveiligingsstrategieën te implementeren.

De integratie van beveiligingstechnieken en theoretische concepten, zoals vermeld in het artikel gepubliceerd in IJITEE, is cruciaal. Het speelt een belangerijke rol 
bij het bijeenbrengen van bestaande theoretische modellen en de praktische uitdagingen van webbeveiliging. Dit onderzoek maakt gebruik van de nieuwste inzichten in cybersecurity 
om bestaande theoretische kaders te evalueren. Dit draagt bij aan de ontwikkeling van vernieuwde theorieën die de complexe problemen van hedendaagse 
cyberdreigingen beter weergeven. Door hedendaagse beveiligingsconcepten te koppelen aan de analyse van penetratietesttools binnen specifieke webomgevingen, 
stimuleert het onderzoek de evolutie van theoretische benaderingen in cybersecurity. Tegelijkertijd levert het essentiële inzichten voor het implementeren van beveiligingsmaatregelen
die overeenkomen met de realiteit van het huidige digitale computercriminaliteit.
Dit proces draagt bij aan het verfijnen en versterken van webbeveiligingspraktijken, gericht op het voldoende te beschermen tegen en reageren op moderne 
cyberdreigingen~\autocite{Nagendran2019}.

\subsection{\IfLanguageName{dutch}{maatschappelijke relevantie}{social relevance}}
Door inzichten uit geavanceerde onderzoeken te integreren, vergroot dit onderzoek de maatschappelijke relevantie van webapplicatiebeveiliging door te verwijzen naar 
twee cruciale bronnen: een studie van ArXiv en onderzoek gepubliceerd in het eJournal of ICT van Akademi Telkom Jakarta. Deze bronnen bieden een uitgebreid overzicht 
van de huidige uitdagingen op het gebied van cybersecurity en de implementatie van geavanceerde beveiligingsmaatregelen en bieden nodige perspectieven die het 
belang van dit onderzoek aan het licht brengen.

Het ArXiv-document presenteert een gedetailleerde analyse van nieuwe cybersecuritydreigingen en de evolutie van aanvalroutes in de context van webapplicaties. 
Door deze bevindingen te integreren, beklemtoont dit onderzoek de toenemende complexiteit van cyberdreigingen en de kritieke noodzaak voor webapplicaties om robuuste 
en dynamische beveiligingsmaatregelen aan te nemen om zich tegen deze evoluerende dreigingen te beschermen. De ArXiv-studie benadrukt het belang van voorop blijven 
in het cybersecurity-spel door beveiligingsprotocollen steeds bij te werken en te versterken, waardoor de bescherming van gebruikersgegevens tegen de nieuwste 
cyberdreigingen gewaarborgd wordt~\autocite{Deng2023}.

Tegelijkertijd gaat de studie gepubliceerd in het eJournal of ICT van Akademi Telkom Jakarta in op specifieke beveiligingsmaatregelen en hun effectiviteit in 
het beschermen van webapplicaties tegen geavanceerde cyberaanvallen. Dit onderzoek draagt bij aan de maatschappelijke impact door praktische kijk te 
bieden in de implementatie van deze beveiligingsmaatregelen en toont hun doeltreffendheid in real-world scenario's aan. Door succesvolle strategieën voor het 
verdedigen tegen cyberaanvallen te presenteren, dient deze studie als een waardevolle bron voor ontwikkelaars, beveiligingsprofessionals en organisaties
die ernaar streven de beveiligingshouding van webapplicaties te versterken ~\autocite{OlivianaZabka2023}.

De maatschappelijke relevantie van dit onderzoek is veelzijdig. Het draagt direct bij aan het verbeteren van de digitale veiligheid en privacy van personen 
en organisaties die afhankelijk zijn van webapplicaties voor verschillende aspecten van hun dagelijks leven. Bij het overwegen van de huidige situatie waarin 
digitale dreigingen ernstige economische en sociale gevolgen kunnen hebben, bieden de inzichten uit de ArXiv- en eJournal of ICT-studies waardevolle informatie. Deze strategieën 
verminderen niet alleen het risico op datalekken en cyberaanvallen, maar bouwen ook vertrouwen op in digitale platforms onder gebruikers.

Samengevat, maatschappelijk levert dit onderzoek een fundamentele bijdrage aan het verbeteren van de digitale veiligheid en het vertrouwen in webapplicaties, die essentieel 
zijn in het dagelijks leven van zowel individuen als organisaties. Door te concentreren op de meest actuele cybersecurity-uitdagingen en oplossingen, 
helpt het bij het maken van nieuwe regels en wetten, en roept op tot strengere beveiligingsstandaarden. Het biedt niet alleen strategieën om het risico 
op datalekken en cyberaanvallen te verminderen, maar versterkt ook het vertrouwen in digitale platforms. Zo draagt het onderzoek bij aan een veiligere 
digitale omgeving in een tijd waarin de digitale dreigingen significante economische en sociale impact kunnen hebben.

%Dit hoofdstuk bevat je literatuurstudie. De inhoud gaat verder op de inleiding, maar zal het onderwerp van de bachelorproef *diepgaand* uitspitten. De bedoeling is dat de lezer na lezing van dit hoofdstuk helemaal op de hoogte is van de huidige stand van zaken (state-of-the-art) in het onderzoeksdomein. Iemand die niet vertrouwd is met het onderwerp, weet nu voldoende om de rest van het verhaal te kunnen volgen, zonder dat die er nog andere informatie moet over opzoeken \autocite{Pollefliet2011}.

%Je verwijst bij elke bewering die je doet, vakterm die je introduceert, enz.\ naar je bronnen. In \LaTeX{} kan dat met het commando \texttt{$\backslash${textcite\{\}}} of \texttt{$\backslash${autocite\{\}}}. Als argument van het commando geef je de ``sleutel'' van een ``record'' in een bibliografische databank in het Bib\LaTeX{}-formaat (een tekstbestand). Als je expliciet naar de auteur verwijst in de zin (narratieve referentie), gebruik je \texttt{$\backslash${}textcite\{\}}. Soms is de auteursnaam niet expliciet een onderdeel van de zin, dan gebruik je \texttt{$\backslash${}autocite\{\}} (referentie tussen haakjes). Dit gebruik je bv.~bij een citaat, of om in het bijschrift van een overgenomen afbeelding, broncode, tabel, enz. te verwijzen naar de bron. In de volgende paragraaf een voorbeeld van elk.

%\textcite{Knuth1998} schreef een van de standaardwerken over sorteer- en zoekalgoritmen. Experten zijn het erover eens dat cloud computing een interessante opportuniteit vormen, zowel voor gebruikers als voor dienstverleners op vlak van informatietechnologie~\autocite{Creeger2009}.

%Let er ook op: het \texttt{cite}-commando voor de punt, dus binnen de zin. Je verwijst meteen naar een bron in de eerste zin die erop gebaseerd is, dus niet pas op het einde van een paragraaf.

%\lipsum[7-20]
