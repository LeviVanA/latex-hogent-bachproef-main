%%=============================================================================
%% Voorwoord
%%=============================================================================

\chapter*{\IfLanguageName{dutch}{Woord vooraf}{Preface}}%
\label{ch:voorwoord}

%% TODO:
%% Het voorwoord is het enige deel van de bachelorproef waar je vanuit je
%% eigen standpunt (``ik-vorm'') mag schrijven. Je kan hier bv. motiveren
%% waarom jij het onderwerp wil bespreken.
%% Vergeet ook niet te bedanken wie je geholpen/gesteund/... heeft

Aan de start van mijn studie Mobile en Enterprise Development was ik mij snel bewust van het belang van cybersecurity in de wereld van softwareontwikkeling. 
Veiligheid wordt steeds relevanter binnen onze samenleving en het is cruciaal dat dit element hand in hand gaat met de ontwikkeling van technologie. 
Met deze context in gedachten koos ik voor mijn bachelorproef een onderwerp dat niet alleen mijn passie in softwareontwikkeling combineert met cybersecurity, 
maar eveneens een essentieel topic in de huidige technologiewereld: het evalueren van de effectiviteit en gebruiksvriendelijkheid van 
penetratietesttools binnen diverse webomgevingen.

Mijn keuze werd verder gedreven door de overtuiging dat een grondige kennis van cybersecurity belangrijk is om mijn vaardigheden als ontwikkelaar naar een hoger niveau te tillen. In het huidige 
tijdperk raken technologische systemen steeds meer verweven met alle aspecten van ons dagelijks leven. Het is daarom cruciaal dat deze systemen niet alleen 
functioneel en efficiënt zijn, maar ook veilig en betrouwbaar. Het besef dat de veiligheid van software daardoor niet alleen de verantwoordelijkheid 
is van security experts, maar van alle IT-professionals, waaronder ontwikkelaars, versterkt enkel nog deze keuze..

Gedurende mijn onderzoek heb ik waardevolle inzichten verkregen hoe verschillende tools presteren in termen van detectie van kwetsbaarheden en de 
gebruiksvriendelijkheid ervan binnen diverse webomgevingen. Dit onderzoek heeft niet alleen mijn kennis en vaardigheden 
verrijkt, maar tevens mijn waardering voor de complexiteit van pentesting tools en het nut ervan voor de beveiliging in softwareontwikkeling versterkt.

Ik wil in dit voorwoord het bedrijf Sinregio en in het bijzonder mijn co-promotor bedanken, Stijn D'Hollander, danken voor zijn bereidheid om te helpen en ondersteuning te bieden  
tijdens dit project. Uiteraard dank ik ook meneer Vertonghen, die als begeleider bereid was om sturing en vorming te 
bieden aan deze bachelorproef. Hun bereidheid om inzichten en constructieve feedback te verschaffen, heeft sterk bijgedragen aan het verbreden en uitdiepen van mijn onderzoek.
Verder wil ik mijn familie bedanken voor hun onvoorwaardelijke steun en geduld gedurende mijn hele studie, niet in het minst tijdens de laatste, 
vaak stressvolle fase. Tot slot wil ik mijn stageplaats, 24Flow, bedanken voor hun hulp en ondersteuning wanneer nodig.

Ik hoop dan ook dat mijn werk bijdraagt aan een groter bewustzijn van het belang van cybersecurity in alle lagen van softwareontwikkeling. 
Moge het lezers inspireren om beveiliging als een noodzakelijke pijler van technologische ontwikkeling te omarmen.


Levi Van Achter

Opdorp, 24 mei 2024
