%%=============================================================================
%% Voorwoord
%%=============================================================================

\chapter*{\IfLanguageName{dutch}{Woord vooraf}{Preface}}%
\label{ch:voorwoord}

%% TODO:
%% Het voorwoord is het enige deel van de bachelorproef waar je vanuit je
%% eigen standpunt (``ik-vorm'') mag schrijven. Je kan hier bv. motiveren
%% waarom jij het onderwerp wil bespreken.
%% Vergeet ook niet te bedanken wie je geholpen/gesteund/... heeft

Aan de start van mijn studie Mobile en Enterprise Development was ik mij al snel bewust van het belang van cybersecurity binnen de wereld van softwareontwikkeling. 
Veiligheid zal ook steeds een relevanter aspect worden binnen onze samenleving en het is cruciaal dat dit element hand in hand gaat met de ontwikkeling van technologie. 
Met deze context in mijn gedachten koos ik voor mijn bachelorproef een onderwerp dat niet alleen mijn interesse in softwareontwikkeling combineert met cybersecurity, 
maar ook een onderwerp dat essentieel is voor de huidige technologiewereld: het evalueren van de effectiviteit en gebruiksvriendelijkheid van verschillende 
penetratietesttools binnen diverse webomgevingen.

Mijn keuze werd gedreven door mijn overtuiging dat een grondige kennis van cybersecurity belangrijk is om mijn vaardigheden als ontwikkelaar verder naar een hoger niveau te tillen. In het huidige 
tijdperk raken technologische systemen steeds meer verweven met alle aspecten van ons dagelijks leven. Het is daarom cruciaal dat deze systemen niet alleen 
functioneel en efficiënt zijn, maar ook veilig en betrouwbaar. Het besef dat de veiligheid van software niet alleen de verantwoordelijkheid 
is van security experts, maar van alle IT-professionals, waaronder ontwikkelaars, heeft mijn interesse in dit onderzoeksgebied tot leven gewekt.

Gedurende mijn onderzoek heb ik waardevolle inzichten verkregen in hoe verschillende tools presteren in termen van detectie van kwetsbaarheden en de 
gebruiksvriendelijkheid ervan binnen verschillende webomgevingen zoals WordPress en Laravel. Dit onderzoek heeft niet alleen mijn kennis en vaardigheden 
verrijkt, maar ook mijn waardering voor de complexiteit van pentesting tools en het belang van beveiliging in softwareontwikkeling versterkt.


Ik wil ook nog mijn co-promotor bedanken, Stijn D'Hollander, voor zijn bereidheid om te helpen en ondersteuning te bieden en tijd vrij te maken 
tijdens dit project. Evenzo ben ik dankbaar voor B. Vertonghen, die als mijn begeleider bereid was om sturing en vormgeving te 
bieden aan deze bachelorproef. Hun bereidheid om inzichten en constructieve feedback te geven, heeft bijgedragen aan het verbreden en verdiepen van mijn onderzoek.
Verder wil ik mijn familie bedanken voor hun onvoorwaardelijke steun en geduld gedurende mijn hele studie, in het bijzonder tijdens deze laatste, 
vaak stressvolle fase.Daarnaast wil ik mijn stageplaats, 24Flow, bedanken voor hun hulp en ondersteuning wanneer nodig.

Met dit voorwoord hoop ik dat mijn werk bijdraagt aan een groter bewustzijn van het belang van cybersecurity in alle lagen van softwareontwikkeling. 
Moge het lezers inspireren om beveiliging als een noodzakelijke pijler van technologische ontwikkeling te omarmen.


Levi Van Achter

Opdorp, 24 mei 2024
